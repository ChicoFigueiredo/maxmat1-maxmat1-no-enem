\section{Questão 150 - Matemática Financeira}

Uma pessoa se interessou em adquirir um produto anunciado em uma loja. Negociou com o gerente e conseguiu comprá-lo a uma taxa de juros compostos de 1\% ao mês. O primeiro pagamento será um mês após a aquisição do produto, e no valor de R\$ 202,00.

O segundo pagamento será efetuado um mês após o primeiro, e terá o valor de R\$ 204,02. Para concretizar a compra, o gerente emitirá uma nota fiscal com o valor do produto à vista negociado com o cliente, correspondendo ao financiamento aprovado.

O valor à vista, em real, que deverá constar na nota fiscal é de

(A) 398,02.

(B) 400,00.

(C) 401,94.

(D) 404,00.

(E) 406,02.

\textbf{Resolução}

Então temos o seguinte fluxo de caixa:

%https://www.mathcha.io
\tikzset{every picture/.style={line width=1.25pt}} %set default line width to 0.75pt        
\noindent \resizebox{\linewidth}{!}{

    \begin{tikzpicture}[x=0.75pt,y=0.75pt,yscale=-1,xscale=1]
    %uncomment if require: \path (0,298); %set diagram left start at 0, and has height of 298
    
    %Straight Lines [id:da741393816524873] 
    \draw [line width=3.75]    (22.4,155.71) -- (678,155.71) ;
    \draw [shift={(685,155.71)}, rotate = 180] [fill={rgb, 255:red, 0; green, 0; blue, 0 }  ][line width=0.08]  [draw opacity=0] (20.54,-9.87) -- (0,0) -- (20.54,9.87) -- cycle    ;
    %Straight Lines [id:da6211040131345571] 
    \draw [line width=3.75]    (22.4,155.71) -- (22.4,17.34) ;
    \draw [shift={(22.4,11.34)}, rotate = 450] [color={rgb, 255:red, 0; green, 0; blue, 0 }  ][line width=3.75]    (25.14,-7.57) .. controls (15.99,-3.21) and (7.61,-0.69) .. (0,0) .. controls (7.61,0.69) and (15.99,3.21) .. (25.14,7.57)   ;
    %Straight Lines [id:da04976474996355207] 
    \draw [line width=3.75]    (190.32,156.39) -- (190.32,228.39) ;
    \draw [shift={(190.32,234.39)}, rotate = 270] [color={rgb, 255:red, 0; green, 0; blue, 0 }  ][line width=3.75]    (25.14,-7.57) .. controls (15.99,-3.21) and (7.61,-0.69) .. (0,0) .. controls (7.61,0.69) and (15.99,3.21) .. (25.14,7.57)   ;
    %Straight Lines [id:da5038339861842813] 
    \draw [line width=3.75]    (359.65,155.03) -- (360.08,234.32) ;
    \draw [shift={(360.11,240.32)}, rotate = 269.69] [color={rgb, 255:red, 0; green, 0; blue, 0 }  ][line width=3.75]    (25.14,-7.57) .. controls (15.99,-3.21) and (7.61,-0.69) .. (0,0) .. controls (7.61,0.69) and (15.99,3.21) .. (25.14,7.57)   ;
    %Straight Lines [id:da3516868561588282] 
    \draw [color={rgb, 255:red, 245; green, 166; blue, 35 }  ,draw opacity=1 ][fill={rgb, 255:red, 245; green, 166; blue, 35 }  ,fill opacity=1 ][line width=3.75]    (176.33,194.34) -- (21.87,194.78) ;
    \draw [shift={(15.87,194.8)}, rotate = 359.84000000000003] [color={rgb, 255:red, 245; green, 166; blue, 35 }  ,draw opacity=1 ][line width=3.75]    (25.14,-7.57) .. controls (15.99,-3.21) and (7.61,-0.69) .. (0,0) .. controls (7.61,0.69) and (15.99,3.21) .. (25.14,7.57)   ;
    %Straight Lines [id:da13727046886374206] 
    \draw [color={rgb, 255:red, 245; green, 166; blue, 35 }  ,draw opacity=1 ][fill={rgb, 255:red, 245; green, 166; blue, 35 }  ,fill opacity=1 ][line width=3]    (361.04,251.13) -- (23.66,250.91) ;
    \draw [shift={(18.66,250.9)}, rotate = 360.03999999999996] [color={rgb, 255:red, 245; green, 166; blue, 35 }  ,draw opacity=1 ][line width=3]    (20.77,-6.25) .. controls (13.2,-2.65) and (6.28,-0.57) .. (0,0) .. controls (6.28,0.57) and (13.2,2.66) .. (20.77,6.25)   ;
    
    % Text Node
    \draw (36.39,11.58) node [anchor=north west][inner sep=0.75pt]  [font=\LARGE]  {$V$};
    % Text Node
    \draw (202.3,206.74) node [anchor=north west][inner sep=0.75pt]  [font=\LARGE,color={rgb, 255:red, 208; green, 2; blue, 27 }  ,opacity=1 ]  {$202,00$};
    % Text Node
    \draw (377.62,202.37) node [anchor=north west][inner sep=0.75pt]  [font=\LARGE,color={rgb, 255:red, 208; green, 2; blue, 27 }  ,opacity=1 ]  {$204,02$};
    % Text Node
    \draw (7.56,205.24) node [anchor=north west][inner sep=0.75pt]  [font=\Large]  {$202,00\ \div \ 1,01 \ $};
    % Text Node
    \draw (111.16,257.77) node [anchor=north west][inner sep=0.75pt]  [font=\Large]  {$204,02\ \div \ 1,01^{2} \ $};
    % Text Node
    \draw (3.6,152.61) node [anchor=north west][inner sep=0.75pt]  [font=\large] [align=left] {0};
    % Text Node
    \draw (185.73,131.24) node [anchor=north west][inner sep=0.75pt]  [font=\large] [align=left] {1};
    % Text Node
    \draw (352.65,132.24) node [anchor=north west][inner sep=0.75pt]  [font=\large] [align=left] {2};
    % Text Node
    \draw (132.65,16.66) node [anchor=north west][inner sep=0.75pt]  [font=\Large]  {$i\ =\ 1\%\ =\ \frac{1}{100} =0,01$};
    % Text Node
    \draw (153.21,56.06) node [anchor=north west][inner sep=0.75pt]  [font=\Large]  {$fator\ de\ capitalização:\ 1+i\ =\ 1\ +\ 0,01\ =\ 1,01$};
    
    
    \end{tikzpicture}

   
}

logo temos que o valor V é a soma das 2 parcelas descontadas no fluxo de caixa:

\begin{eqnarray*}
V &=& \dfrac{202,00}{1,01} + \dfrac{204,02}{1,01^2} \\
  &=& \dfrac{202,00}{1,01} + \dfrac{204,02}{1,0201} \\
  &=& \dfrac{20200}{101} + \dfrac{2040200}{10201} \\
  &=& 200 + 200 \\  
  &=& 400 \\
\end{eqnarray*}

\textbf{Resposta:} A nota fiscal deverá ser ser preenchda com o valor de R\$ 400,00. Alternativa (B)

\textbf{Rascunho}

\opmul[decimalsepsymbol={,},displayintermediary=all]{1.01}{1.01}\flexquad{3}
\opdiv[decimalsepsymbol={,},displayintermediary=all]{202}{1.01}\flexquad{3}
\opset{strikedecimalsepsymbol={\rlap{,}\rule[-1pt]{3pt}{0.4pt}}}
\opdiv[decimalsepsymbol={,},shiftdecimalsep=both,displayintermediary=all]{204.02}{1.0201}\quad


\begin{center}
    \href{https://youtu.be/S22SHYt4n-o}{
        \qrcode{https://youtu.be/S22SHYt4n-o}
    }\\
    {\scriptsize Resolução: \url{https://youtu.be/S22SHYt4n-o}}
\end{center}