\chapter{ENEM 2019 - Caderno Azul}

\section{Questão 150 - Matemática Financeira}

Uma pessoa se interessou em adquirir um produto anunciado em uma loja. Negociou com o gerente e conseguiu comprá-lo a uma taxa de juros compostos de 1\% ao mês. O primeiro pagamento será um mês após a aquisição do produto, e no valor de R\$ 202,00.

O segundo pagamento será efetuado um mês após o primeiro, e terá o valor de R\$ 204,02. Para concretizar a compra, o gerente emitirá uma nota fiscal com o valor do produto à vista negociado com o cliente, correspondendo ao financiamento aprovado.

O valor à vista, em real, que deverá constar na nota fiscal é de

(A) 398,02.

(B) 400,00.

(C) 401,94.

(D) 404,00.

(E) 406,02.

\textbf{Resolução}

Então temos o seguinte fluxo de caixa:

%https://www.mathcha.io
\tikzset{every picture/.style={line width=0.75pt}} %set default line width to 0.75pt        
\begin{tikzpicture}[x=0.75pt,y=0.75pt,yscale=-1,xscale=1]
    %uncomment if require: \path (0,219); %set diagram left start at 0, and has height of 219
    
    %Straight Lines [id:da741393816524873] 
    \draw    (20,116) -- (490.5,116) ;
    \draw [shift={(493.5,116)}, rotate = 180] [fill={rgb, 255:red, 0; green, 0; blue, 0 }  ][line width=0.08]  [draw opacity=0] (8.93,-4.29) -- (0,0) -- (8.93,4.29) -- cycle    ;
    %Straight Lines [id:da6211040131345571] 
    \draw    (20,116) -- (20,12.5) ;
    \draw [shift={(20,10.5)}, rotate = 450] [color={rgb, 255:red, 0; green, 0; blue, 0 }  ][line width=0.75]    (10.93,-3.29) .. controls (6.95,-1.4) and (3.31,-0.3) .. (0,0) .. controls (3.31,0.3) and (6.95,1.4) .. (10.93,3.29)   ;
    %Straight Lines [id:da04976474996355207] 
    \draw    (140,116.5) -- (140,171.5) ;
    \draw [shift={(140,173.5)}, rotate = 270] [color={rgb, 255:red, 0; green, 0; blue, 0 }  ][line width=0.75]    (10.93,-3.29) .. controls (6.95,-1.4) and (3.31,-0.3) .. (0,0) .. controls (3.31,0.3) and (6.95,1.4) .. (10.93,3.29)   ;
    %Straight Lines [id:da5038339861842813] 
    \draw    (261,115.5) -- (261.32,175.83) ;
    \draw [shift={(261.33,177.83)}, rotate = 269.69] [color={rgb, 255:red, 0; green, 0; blue, 0 }  ][line width=0.75]    (10.93,-3.29) .. controls (6.95,-1.4) and (3.31,-0.3) .. (0,0) .. controls (3.31,0.3) and (6.95,1.4) .. (10.93,3.29)   ;
    %Straight Lines [id:da3516868561588282] 
    \draw [color={rgb, 255:red, 245; green, 166; blue, 35 }  ,draw opacity=1 ][fill={rgb, 255:red, 245; green, 166; blue, 35 }  ,fill opacity=1 ]   (130,143.5) -- (17.33,143.83) ;
    \draw [shift={(15.33,143.83)}, rotate = 359.83000000000004] [color={rgb, 255:red, 245; green, 166; blue, 35 }  ,draw opacity=1 ][line width=0.75]    (10.93,-3.29) .. controls (6.95,-1.4) and (3.31,-0.3) .. (0,0) .. controls (3.31,0.3) and (6.95,1.4) .. (10.93,3.29)   ;
    %Straight Lines [id:da13727046886374206] 
    \draw [color={rgb, 255:red, 245; green, 166; blue, 35 }  ,draw opacity=1 ][fill={rgb, 255:red, 245; green, 166; blue, 35 }  ,fill opacity=1 ]   (262,185) -- (19.33,184.83) ;
    \draw [shift={(17.33,184.83)}, rotate = 360.03999999999996] [color={rgb, 255:red, 245; green, 166; blue, 35 }  ,draw opacity=1 ][line width=0.75]    (10.93,-3.29) .. controls (6.95,-1.4) and (3.31,-0.3) .. (0,0) .. controls (3.31,0.3) and (6.95,1.4) .. (10.93,3.29)   ;
    
    % Text Node
    \draw (28,7.9) node [anchor=north west][inner sep=0.75pt]    {$V$};
    % Text Node
    \draw (149,154.9) node [anchor=north west][inner sep=0.75pt]    {$202,00$};
    % Text Node
    \draw (270,153.9) node [anchor=north west][inner sep=0.75pt]    {$204,02$};
    % Text Node
    \draw (31,146.9) node [anchor=north west][inner sep=0.75pt]  [font=\footnotesize]  {$202,00\ \div \ 1,01 \ $};
    % Text Node
    \draw (35,190.4) node [anchor=north west][inner sep=0.75pt]  [font=\footnotesize]  {$204,02\ \div \ 1,01^{2} \ $};
    % Text Node
    \draw (15,119.5) node [anchor=north west][inner sep=0.75pt]  [font=\tiny] [align=left] {0};
    % Text Node
    \draw (128,118.5) node [anchor=north west][inner sep=0.75pt]  [font=\tiny] [align=left] {1};
    % Text Node
    \draw (248,118.5) node [anchor=north west][inner sep=0.75pt]  [font=\tiny] [align=left] {2};
    % Text Node
    \draw (430,11.4) node [anchor=north west][inner sep=0.75pt]  [font=\scriptsize]  {$i\ =\ 1\%\ =\ \frac{1}{100} =0,01$};
    % Text Node
    \draw (425,41.4) node [anchor=north west][inner sep=0.75pt]  [font=\scriptsize]  {$fator\ de\ capitalização:\ 1+1\ =\ 1\ +\ 0,01\ =\ 1,01$};

\end{tikzpicture}

logo temos que o valor V é a soma das 2 parcelas descontadas no fluxo de caixa:

\begin{eqnarray*}
V &=& \dfrac{202,00}{1,01} + \dfrac{204,02}{1,01^2} \\
  &=& \dfrac{202,00}{1,01} + \dfrac{204,02}{1,0201} \\
  &=& \dfrac{20200}{101} + \dfrac{2040200}{10201} \\
  &=& 200 + 200 \\  
  &=& 400 \\
\end{eqnarray*}

A nota fiscal deverá ser ser preenchda com o valor de R\$ 400,00. Alternativa (B)

\textbf{Rascunho}

\opmul[decimalsepsymbol={,},displayintermediary=all]{1.01}{1.01}