\chapter{ENEM 2019 - Caderno Azul}

\section{Questão 150 - Matemática Financeira}

Uma pessoa se interessou em adquirir um produto anunciado em uma loja. Negociou com o gerente e conseguiu comprá-lo a uma taxa de juros compostos de 1\% ao mês. O primeiro pagamento será um mês após a aquisição do produto, e no valor de R\$ 202,00.

O segundo pagamento será efetuado um mês após o primeiro, e terá o valor de R\$ 204,02. Para concretizar a compra, o gerente emitirá uma nota fiscal com o valor do produto à vista negociado com o cliente, correspondendo ao financiamento aprovado.

O valor à vista, em real, que deverá constar na nota fiscal é de

(A) 398,02.

(B) 400,00.

(C) 401,94.

(D) 404,00.

(E) 406,02.

\textbf{Resolução}

Então temos o seguinte fluxo de caixa:

%https://www.mathcha.io
\tikzset{every picture/.style={line width=0.75pt}} %set default line width to 0.75pt        
\begin{tikzpicture}[x=0.75pt,y=0.75pt,yscale=-1,xscale=1]
    %uncomment if require: \path (0,219); %set diagram left start at 0, and has height of 219
    
    %Straight Lines [id:da741393816524873] 
    \draw    (20,116) -- (490.5,116) ;
    \draw [shift={(493.5,116)}, rotate = 180] [fill={rgb, 255:red, 0; green, 0; blue, 0 }  ][line width=0.08]  [draw opacity=0] (8.93,-4.29) -- (0,0) -- (8.93,4.29) -- cycle    ;
    %Straight Lines [id:da6211040131345571] 
    \draw    (20,116) -- (20,12.5) ;
    \draw [shift={(20,10.5)}, rotate = 450] [color={rgb, 255:red, 0; green, 0; blue, 0 }  ][line width=0.75]    (10.93,-3.29) .. controls (6.95,-1.4) and (3.31,-0.3) .. (0,0) .. controls (3.31,0.3) and (6.95,1.4) .. (10.93,3.29)   ;
    %Straight Lines [id:da04976474996355207] 
    \draw    (140,116.5) -- (140,171.5) ;
    \draw [shift={(140,173.5)}, rotate = 270] [color={rgb, 255:red, 0; green, 0; blue, 0 }  ][line width=0.75]    (10.93,-3.29) .. controls (6.95,-1.4) and (3.31,-0.3) .. (0,0) .. controls (3.31,0.3) and (6.95,1.4) .. (10.93,3.29)   ;
    %Straight Lines [id:da5038339861842813] 
    \draw    (261,115.5) -- (261.32,175.83) ;
    \draw [shift={(261.33,177.83)}, rotate = 269.69] [color={rgb, 255:red, 0; green, 0; blue, 0 }  ][line width=0.75]    (10.93,-3.29) .. controls (6.95,-1.4) and (3.31,-0.3) .. (0,0) .. controls (3.31,0.3) and (6.95,1.4) .. (10.93,3.29)   ;
    %Straight Lines [id:da3516868561588282] 
    \draw [color={rgb, 255:red, 245; green, 166; blue, 35 }  ,draw opacity=1 ][fill={rgb, 255:red, 245; green, 166; blue, 35 }  ,fill opacity=1 ]   (130,143.5) -- (17.33,143.83) ;
    \draw [shift={(15.33,143.83)}, rotate = 359.83000000000004] [color={rgb, 255:red, 245; green, 166; blue, 35 }  ,draw opacity=1 ][line width=0.75]    (10.93,-3.29) .. controls (6.95,-1.4) and (3.31,-0.3) .. (0,0) .. controls (3.31,0.3) and (6.95,1.4) .. (10.93,3.29)   ;
    %Straight Lines [id:da13727046886374206] 
    \draw [color={rgb, 255:red, 245; green, 166; blue, 35 }  ,draw opacity=1 ][fill={rgb, 255:red, 245; green, 166; blue, 35 }  ,fill opacity=1 ]   (262,185) -- (19.33,184.83) ;
    \draw [shift={(17.33,184.83)}, rotate = 360.03999999999996] [color={rgb, 255:red, 245; green, 166; blue, 35 }  ,draw opacity=1 ][line width=0.75]    (10.93,-3.29) .. controls (6.95,-1.4) and (3.31,-0.3) .. (0,0) .. controls (3.31,0.3) and (6.95,1.4) .. (10.93,3.29)   ;
    
    % Text Node
    \draw (28,7.9) node [anchor=north west][inner sep=0.75pt]    {$V$};
    % Text Node
    \draw (149,154.9) node [anchor=north west][inner sep=0.75pt]    {$202,00$};
    % Text Node
    \draw (270,153.9) node [anchor=north west][inner sep=0.75pt]    {$204,02$};
    % Text Node
    \draw (31,146.9) node [anchor=north west][inner sep=0.75pt]  [font=\footnotesize]  {$202,00\ \div \ 1,01 \ $};
    % Text Node
    \draw (35,190.4) node [anchor=north west][inner sep=0.75pt]  [font=\footnotesize]  {$204,02\ \div \ 1,01^{2} \ $};
    % Text Node
    \draw (15,119.5) node [anchor=north west][inner sep=0.75pt]  [font=\tiny] [align=left] {0};
    % Text Node
    \draw (128,118.5) node [anchor=north west][inner sep=0.75pt]  [font=\tiny] [align=left] {1};
    % Text Node
    \draw (248,118.5) node [anchor=north west][inner sep=0.75pt]  [font=\tiny] [align=left] {2};
    % Text Node
    \draw (430,11.4) node [anchor=north west][inner sep=0.75pt]  [font=\scriptsize]  {$i\ =\ 1\%\ =\ \frac{1}{100} =0,01$};
    % Text Node
    \draw (425,41.4) node [anchor=north west][inner sep=0.75pt]  [font=\scriptsize]  {$fator\ de\ capitalização:\ 1+1\ =\ 1\ +\ 0,01\ =\ 1,01$};

\end{tikzpicture}

logo temos que o valor V é a soma das 2 parcelas descontadas no fluxo de caixa:

\begin{eqnarray*}
V &=& \dfrac{202,00}{1,01} + \dfrac{204,02}{1,01^2} \\
  &=& \dfrac{202,00}{1,01} + \dfrac{204,02}{1,0201} \\
  &=& \dfrac{20200}{101} + \dfrac{2040200}{10201} \\
  &=& 200 + 200 \\  
  &=& 400 \\
\end{eqnarray*}

A nota fiscal deverá ser ser preenchda com o valor de R\$ 400,00. Alternativa (B)

\textbf{Rascunho}

\opmul[decimalsepsymbol={,},displayintermediary=all]{1.01}{1.01}\quad
\opdiv[decimalsepsymbol={,},displayintermediary=all]{202}{1.01}\quad
\opset{strikedecimalsepsymbol={\rlap{,}\rule[-1pt]{3pt}{0.4pt}}}
\opdiv[decimalsepsymbol={,},shiftdecimalsep=both,displayintermediary=all]{204.02}{1.0201}\quad




\section{Questão 154 - Logaritmos, unidades de medida, interpretação tabela}

Charles Richter e Beno Gutenberg desenvolveram a escala Richter, que mede a magnitude de um terremoto. Essa escala pode variar de 0 a 10, com possibilidades de valores maiores. O quadro mostra a escala de magnitude local $ (M_{s}) $ de um terremoto que é utilizada para descrevê-lo.


% Please add the following required packages to your document preamble:
% \usepackage[table,xcdraw]{xcolor}
% If you use beamer only pass "xcolor=table" option, i.e. \documentclass[xcolor=table]{beamer}
% https://www.tablesgenerator.com
\begin{center}
    \begin{tabular}{cc}
        \rowcolor[HTML]{BFBFBF} 
        {\color[HTML]{000000} Descrição} & {\color[HTML]{000000}
            \begin{tabular}[c]{@{}c@{}}
                Magnitude   local $ (M_{s}) $\\  $ (\mu m \cdot Hz) $\end{tabular}} \\
                {\color[HTML]{000000} Pequeno}   & {\color[HTML]{000000} $ 0 \leqslant M_{s} \leqslant 3,9 $}                                                                \\
                {\color[HTML]{000000} Ligeiro}   & {\color[HTML]{000000} $ 4,0 \leqslant M_{s} \leqslant 4,9 $}                                                              \\
                {\color[HTML]{000000} Moderado}  & {\color[HTML]{000000} $ 5,0 \leqslant M_{s} \leqslant 5,9 $}                                                              \\
                {\color[HTML]{000000} Grande}    & {\color[HTML]{000000} $ 6,0 \leqslant M_{s} \leqslant 9,9 $}                                                              \\
                {\color[HTML]{000000} Extremo}   & {\color[HTML]{000000} $ M_{s} \geqslant   10,0 $}                                                                
    \end{tabular}
\end{center}



Para se calcular a magnitude local, usa-se a fórmula $ M_{s} = 3,30 + log(A \cdot f)$, em que $ A $ representa a amplitude máxima da onda registrada por um sismógrafo em micrômetro ($\mu m$) e $ f $ representa a frequência da onda, em hertz (Hz). Ocorreu um terremoto com amplitude máxima de $ 2 000 \mu m $ e frequência de $ 0,2 Hz $. 

{\scriptsize 
    \begin{flushright}
        Disponível em: http://cejarj.cecierj.edu.br. Acesso em: 1 fev. 2015 (adaptado). 
    \end{flushright}
}

Utilize 0,3 como aproximação para log 2. 

De acordo com os dados fornecidos, o terremoto ocorrido pode ser descrito como

(A) Pequeno.

(B) Ligeiro.

(C) Moderado.

(D) Grande.

(E) Extremo.



\textbf{Resolução}

Temos um caso simples de substituição de variáveis, e bom uso das propriedades de logaritmos:


\begin{eqnarray*}
    M_{s} 	&=& 3,30 + log(A \cdot f) \\
            &=& 3,30 + log(2000 \cdot 0,2) \\
            &=& 3,30 + log(400) \\
            &=& 3,30 + log(4 \cdot 100) \\
            &=& 3,30 + log(2^{2} \cdot 10^{2}) \\
            &=& 3,30 + log(2^{2}) + log(10^{2}) \\
            &=& 3,30 + 2 \cdot log(2) + 2 \cdot log(10) \\
            &=& 3,30 + 2 \cdot 0,3 + 2 \cdot 1 \\
            &=& 3,30 + 0,6 + 2 \\
            &=& 5,9 \\
\end{eqnarray*}


\textbf{Rascunho}

\opmul[decimalsepsymbol={,},displayintermediary=all]{2000}{0.2}\quad
\opmul[decimalsepsymbol={,},displayintermediary=all]{2}{0.3}\quad
\opadd[decimalsepsymbol={,},displayintermediary=all]{3.3}{0.6}{2}\quad

