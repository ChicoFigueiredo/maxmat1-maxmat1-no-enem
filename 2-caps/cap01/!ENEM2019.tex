\chapter{ENEM 2019 - Caderno Azul}

\section{Questão 150 - Matemática Financeira}

Uma pessoa se interessou em adquirir um produto anunciado em uma loja. Negociou com o gerente e conseguiu comprá-lo a uma taxa de juros compostos de 1\% ao mês. O primeiro pagamento será um mês após a aquisição do produto, e no valor de R\$ 202,00.

O segundo pagamento será efetuado um mês após o primeiro, e terá o valor de R\$ 204,02. Para concretizar a compra, o gerente emitirá uma nota fiscal com o valor do produto à vista negociado com o cliente, correspondendo ao financiamento aprovado.

O valor à vista, em real, que deverá constar na nota fiscal é de

(A) 398,02.

(B) 400,00.

(C) 401,94.

(D) 404,00.

(E) 406,02.

\textbf{Resolução}

temos: 

$ i = 1\% = \frac{1}{100} = 0,01 $

fator de capitalização: $ 1 + i = 1 + 0,01 \Longrightarrow 1 + i = 1,01 $

\begin{figure}[ht] \label{fluxo}
    \unitlength=1mm \caption{Fluxo de caixa}
    \centering{\fbox{\begin{picture}(100,50)
    \drawline(10,25)(90,25)
    
    %VPL
    \put(10,25){\vector(0, 1){20}}
    

    \put(30,25){\vector(0,-1){15}}
    \put(60,25){\vector(0,-1){15}}

    %\put(20,25){\vector(0,-1){15}}
    %\put(30,25){\vector(0,-1){15}}
    %\put(40,25){\vector(0,-1){15}}
    %\put(50,25){\vector(0,-1){15}}
    %\put(70,25){\vector(0,-1){15}}
    %\put(80,25){\vector(0,-1){15}}
    %\put(90,25){\vector(0,-1){15}}
    
    \put( 6,40){P} \put(16,20){1}
    \put(26,20){R} \put(36,20){2}
    %\put(46,20){R} \put(57,20){...}
    %\put(76,20){R} \put(86,20){R}
    
    \put(50,40){i=1\% / 1 + i = 1,01} \put(5,27){0}
    
    \put(29,27){1} 
    \put(59,27){2}
    %\put(19,27){1} 
    %\put(29,27){2}
    %\put(39,27){3} 
    %\put(49,27){4}
    %\put(67,27){n-2} 
    %\put(77,27){n-1} 
    %\put(89,27){n}
    \end{picture}}}
\end{figure}

