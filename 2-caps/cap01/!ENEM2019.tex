\chapter{ENEM 2019 - Caderno Azul}

\section{Questão 150 - Matemática Financeira}

Uma pessoa se interessou em adquirir um produto anunciado em uma loja. Negociou com o gerente e conseguiu comprá-lo a uma taxa de juros compostos de 1\% ao mês. O primeiro pagamento será um mês após a aquisição do produto, e no valor de R\$ 202,00.
O segundo pagamento será efetuado um mês após o primeiro, e terá o valor de R\$ 204,02. Para concretizar a compra, o gerente emitirá uma nota fiscal com o valor do produto à vista negociado com o cliente, correspondendo ao financiamento aprovado.
O valor à vista, em real, que deverá constar na nota fiscal é de
(A) 398,02.
(B) 400,00.
(C) 401,94.
(D) 404,00.
(E) 406,02.
