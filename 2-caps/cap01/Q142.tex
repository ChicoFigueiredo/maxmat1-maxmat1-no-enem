lLzEawlb-xw\section{Questão 142 - Geometria, Circulo, Área}

Em um condomínio, uma área pavimentada, que tem a forma de um círculo com diâmetro medindo $ 6 m $, é cercada por grama. A administração do condomínio deseja ampliar essa área, mantendo seu formato circular, e aumentando, em $ 8 m $, o diâmetro dessa região, mantendo o revestimento da parte já existente. O condomínio dispõe, em estoque, de material suficiente para pavimentar mais $ 100 m^2 $ de área. O síndico do condomínio irá avaliar se esse material disponível será suficiente para pavimentar a região a ser ampliada.

Utilize 3 como aproximação para $ \pi $.

A conclusão correta a que o síndico deverá chegar, considerando a nova área a ser pavimentada, é a de que o material disponível em estoque 

(A)  	será suficiente, pois a área da nova região a ser pavimentada mede $21 m^2$.

(B)  	será suficiente, pois a área da nova região a ser pavimentada mede $24 m^2$.

(C)  	será suficiente, pois a área da nova região a ser pavimentada mede $48 m^2$.

(D) 	não será suficiente, pois a área da nova região a ser pavimentada mede $108 m^2$.

(E) 	não será suficiente, pois a área da nova região a ser pavimentada mede $120 m^2$.

\textbf{Resolução}

\textbf{Rascunho}

\opmul[decimalsepsymbol={,},displayintermediary=all]{1.01}{1.01}\flexquad{3}
\opdiv[decimalsepsymbol={,},displayintermediary=all]{202}{1.01}\flexquad{3}
\opset{strikedecimalsepsymbol={\rlap{,}\rule[-1pt]{3pt}{0.4pt}}}
\opdiv[decimalsepsymbol={,},shiftdecimalsep=both,displayintermediary=all]{204.02}{1.0201}\quad


\begin{center}
    \href{https://youtu.be/lLzEawlb-xw}{
        \qrcode{https://youtu.be/lLzEawlb-xw}
    }\\
    Resolução: \url{https://youtu.be/lLzEawlb-xw}
\end{center}
