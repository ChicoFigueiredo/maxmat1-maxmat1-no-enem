lLzEawlb-xw\section{Questão 142 - Geometria, Circulo, Área}

Em um condomínio, uma área pavimentada, que tem a forma de um círculo com diâmetro medindo $ 6 m $, é cercada por grama. A administração do condomínio deseja ampliar essa área, mantendo seu formato circular, e aumentando, em $ 8 m $, o diâmetro dessa região, mantendo o revestimento da parte já existente. O condomínio dispõe, em estoque, de material suficiente para pavimentar mais $ 100 m^2 $ de área. O síndico do condomínio irá avaliar se esse material disponível será suficiente para pavimentar a região a ser ampliada.

Utilize 3 como aproximação para $ \pi $.

A conclusão correta a que o síndico deverá chegar, considerando a nova área a ser pavimentada, é a de que o material disponível em estoque 

(A)  	será suficiente, pois a área da nova região a ser pavimentada mede $21 m^2$.

(B)  	será suficiente, pois a área da nova região a ser pavimentada mede $24 m^2$.

(C)  	será suficiente, pois a área da nova região a ser pavimentada mede $48 m^2$.

(D) 	não será suficiente, pois a área da nova região a ser pavimentada mede $108 m^2$.

(E) 	não será suficiente, pois a área da nova região a ser pavimentada mede $120 m^2$.

\textbf{Resolução}

\noindent \resizebox{.5\textwidth}{!}{
    \begin{tikzpicture}[>=latex,scale=.5]
    
    %\node (circle) [draw, circle, minimum size = 7cm, label=above:$M$] at (0, 0) {};
    %\node (circle) [draw, circle, minimum size = 3cm, label=above:$m$] at (0, 0) {};
    
    %\node (circle) [draw, circle, raduis = 7, label=above:$M$] at (0, 0) {};
    
    \draw[fill=green!10,even odd rule]  (0,0) circle (7)
    (0,0) circle (3);
    
    \draw [blue] circle [radius=7];  
    \draw [blue] circle [radius=3];
    
    \draw [<->](-7,9)--node[above]{$D = 6 + 8 = 14 \Rightarrow R = 7 $} (7,9);
    \draw [dotted,-](-7,9)--(-7,0);
    \draw [dotted,-](7,9)--(7,0);
    
    \draw [<->](-3,8)--node[above]{$d = 6 \Rightarrow r = 3 $} (3,8);
    \draw [dotted,-](-3,8)--(-3,0);
    \draw [dotted,-](3,8)--(3,0);
    
    \draw [->](0,0)--node[above right]{$r = 3$} (-45:3);
    
    \draw [<->](0,0)--node[below left=.4 and 0]{$R = 7$} (-60:7);
    
    \draw [->](-120:5)--(-120:10)node [below right=.0 and -1]{Área Coroa: $A = \pi R^2 - \pi r^2$ };
    
    
    %\draw [->](-5:-5)--(-7,-7);
    
    \end{tikzpicture}
}


\begin{eqnarray*}
    A &=&  \pi R^2 - \pi r^2 \\
      &=&  \pi ( R^2 - r^2 ) \\
      &=&  3 ( 7^2 - 3^2 ) \\
      &=&  3 ( 49 - 9 ) \\
      &=&  3 \cdot 40 \\
      & &    \\
    A &=&  120 m^2 
\end{eqnarray*}

em virtude de haver apenas $ 100m^2 $ de material, não será suficiente, pois a área da nova região a ser pavimentada mede $120 m^2$, correspondendo a letra (E).

\textbf{Rascunho}
\opmul[decimalsepsymbol={,},displayintermediary=all]{40}{3}\flexquad{3}
\opsub[decimalsepsymbol={,},displayintermediary=all]{49}{9}\flexquad{3}

\begin{center}
    \href{https://youtu.be/lLzEawlb-xw}{
        \qrcode{https://youtu.be/lLzEawlb-xw}
    }\\
    Resolução: \url{https://youtu.be/lLzEawlb-xw}
\end{center}
