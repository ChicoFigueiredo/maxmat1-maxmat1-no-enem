\section{Questão 151 Laranja Ledor - Interpretação de gráficos}

Na anestesia peridural, como a usada nos partos, o médico anestesista precisa introduzir uma agulha nas costas do paciente, que atravessará várias camadas de tecido até chegar a uma região estreita, chamada espaço epidural, que envolve a medula espinhal. A agulha é usada para injetar um líquido anestésico, e a força que deve ser aplicada à agulha para fazê-la  avançar através dos tecidos é variável.

A figura é um gráfico do módulo F da força (em newton) em função do deslocamento x da ponta da agulha (em milímetro) durante uma anestesia peridural típica. 

\textbf{Descrição da figura:} A figura mostra um gráfico em um plano cartesiano. O eixo x representa o deslocamento da ponta da agulha, em milímetro, e o eixo y representa o módulo da força, em Newton. O gráfico é representado por uma linha poligonal com oito segmentos,

Segmento AB: do ponto A(0 ; 0) ao ponto B(8 ; 12).

Segmento BC: do ponto B(8 ; 12) ao ponto C(12 ; 6).

Segmento CD: do ponto C(12 ; 6) ao ponto D(18 ; 8).

Segmento DE: do ponto D(18 ; 8) ao ponto E(20 ; 7).

Segmento EF: do ponto E(20 ; 7) ao ponto F(26 ; 7).

Segmento FG: do ponto F(26 ; 7) ao ponto G(28 ; 12).

Segmento GH: do ponto G(28 ; 12) ao ponto H(30 ; 12).

Segmento HI: do ponto H(30 ; 12) ao ponto I(32 ; 3).


Considere que a velocidade de penetração da agulha deva ser a mesma durante a aplicação da anestesia e que a força aplicada à agulha pelo médico anestesista em cada ponto deve ser proporcional à resistência naquele ponto.

Com base nas informações apresentadas, a maior resistência à força aplicada observa-se ao longo do segmento

(A) AB.

(B) FG.

(C) EF.

(D) GH.

(E) HI.


\textbf{Resolução}

A figura descrita pode ser desenhada da seguinte forma:

\noindent \resizebox{\linewidth}{!}{
    \begin{tikzpicture}[scale=.5,y=1cm, x=.7cm,font=\sffamily]
        %axis
        \draw [->] (0,0) -- coordinate (x axis mid) (32,0);
        \draw [->] (0,0) -- coordinate (y axis mid) (0,14);
        %ticks
        \foreach \x in {0,5,...,32}
        \draw (\x,1pt) -- (\x,-3pt) node[anchor=north] {\x};
        \foreach \y in {0,...,14}
        \draw (1pt,\y) -- (-3pt,\y)  node[anchor=east] {\y}; 
        %labels      
        \node[below=0.8cm] at (x axis mid) {x = deslocamento da agulha [mm]};
        \node[rotate=90, above left=0.8cm and .8cm] at (y axis mid) {y = módulo da força [N]};
        
        %plots
        \node[label={-45:{A( 0 ; 1 )}},circle,fill,inner sep=2pt] at (0,0) (A) {};
        \node[label={90:{B( 8 ; 12 )}},circle,fill,inner sep=2pt] at (8,12) (B) {};
        \node[label={-90:{C( 12 ; 6 )}},circle,fill,inner sep=2pt] at (12, 6) (C) {};
        \node[label={90:{D( 18 ; 8 )}},circle,fill,inner sep=2pt] at (18, 8) (D) {};
        \node[label={-120:{E( 20 ; 7 )}},circle,fill,inner sep=2pt] at (20, 7) (E) {};
        \node[label={120:{F( 26 ; 7 )}},circle,fill,inner sep=2pt] at (26, 7) (F) {};
        \node[label={120:{G( 28 ; 12 )}},circle,fill,inner sep=2pt] at (28, 12) (G) {};
        \node[label={46:{H( 28 ; 12 )}},circle,fill,inner sep=2pt] at (30, 12) (H) {};
        \node[label={0:{I( 32 ; 3 )}},circle,fill,inner sep=2pt] at (32, 3) (I) {};

        
        \draw [ultra thick	] (A) -- (B) -- (C) -- (D) -- (E) -- (F) -- (G) -- (H) -- (I) ;
    \end{tikzpicture}
}

Vemos claramente que estamos falando do trabalho realizado por uma força em relação ao seu deslocamento.

Para determinar qual o maior trabalho realizado, vamos calcular a área abaixo de cada segmento das questões:


\noindent \resizebox{\linewidth}{!}{
        \begin{tikzpicture}[scale=.5,y=1cm, x=.7cm,font=\sffamily]
    %axis
    \draw [->] (0,0) -- coordinate (x axis mid) (32,0);
    \draw [->] (0,0) -- coordinate (y axis mid) (0,14);
    %ticks
    \foreach \x in {0,5,...,32}
    \draw (\x,1pt) -- (\x,-3pt) node[anchor=north] {\x};
    \foreach \y in {0,...,14}
    \draw (1pt,\y) -- (-3pt,\y)  node[anchor=east] {\y}; 
    %labels      
    \node[below=0.8cm] at (x axis mid) {x = deslocamento da agulha [mm]};
    \node[rotate=90, above left=0.8cm and .8cm] at (y axis mid) {y = módulo da força [N]};
    
    
    %plots
    \node[label={-45:{A( 0 ; 1 )}},circle,fill,inner sep=2pt] at (0,0) (A) {};
    \node[label={90:{B( 8 ; 12 )}},circle,fill,inner sep=2pt] at (8,12) (B) {};
    \node[label={-90:{C( 12 ; 6 )}},circle,fill,inner sep=2pt] at (12, 6) (C) {};
    \node[label={90:{D( 18 ; 8 )}},circle,fill,inner sep=2pt] at (18, 8) (D) {};
    \node[label={-120:{E( 20 ; 7 )}},circle,fill,inner sep=2pt] at (20, 7) (E) {};
    \node[label={120:{F( 26 ; 7 )}},circle,fill,inner sep=2pt] at (26, 7) (F) {};
    \node[label={120:{G( 28 ; 12 )}},circle,fill,inner sep=2pt] at (28, 12) (G) {};
    \node[label={46:{H( 28 ; 12 )}},circle,fill,inner sep=2pt] at (30, 12) (H) {};
    \node[label={0:{I( 32 ; 3 )}},circle,fill,inner sep=2pt] at (32, 3) (I) {};
    
    \filldraw [thick,red,fill=green,opacity=.3] (A.center) -- (B.center) -- (8, 0) -- (A.center); 
    \filldraw [thick,red,fill=green,opacity=.3] (F.center) -- (G.center) -- (28, 0) -- (26, 0) -- (F.center); 
    \filldraw [thick,red,fill=green,opacity=.3] (E.center) -- (F.center) -- (26, 0) -- (20, 0) -- (E.center); 
    \filldraw [thick,red,fill=green,opacity=.3] (G.center) -- (H.center) -- (30, 0) -- (28, 0) -- (G.center); 
    \filldraw [thick,red,fill=green,opacity=.3] (H.center) -- (I.center) -- (32, 0) -- (30, 0) -- (H.center); 
    
    \draw [ultra thick] (A) -- (B) -- (C) -- (D) -- (E) -- (F) -- (G) -- (H) -- (I) ;
    
    
    \end{tikzpicture}   
}

\begin{itemize}
    \item AB $ \Rightarrow $ Triângulo Retângulo base 8 e altura 12 $ \Rightarrow T = \dfrac{B \cdot h}{2} = \dfrac{12 \cdot 8}{2} = 6 \cdot 2 = 12 N/mm  $ 
    \item FG $ \Rightarrow $ Trapézio base maior 12, base menor 7 e altura $ 26 - 28 = 2 \Rightarrow A = \dfrac{(B + b) \cdot h}{2} = \dfrac{(12 + 7) \cdot 2}{2} = 12 + 7 = 19 N/mm $
    
    
\end{itemize}

\textbf{Rascunho}

\opmul[decimalsepsymbol={,},displayintermediary=all]{1.01}{1.01}\flexquad{3}
\opdiv[decimalsepsymbol={,},displayintermediary=all]{202}{1.01}\flexquad{3}
\opset{strikedecimalsepsymbol={\rlap{,}\rule[-1pt]{3pt}{0.4pt}}}
\opdiv[decimalsepsymbol={,},shiftdecimalsep=both,displayintermediary=all]{204.02}{1.0201}\quad


\begin{center}
    \href{https://youtu.be/}{
        \qrcode{https://youtu.be/}
    }\\
    Resolução: \url{https://youtu.be/}
\end{center}
