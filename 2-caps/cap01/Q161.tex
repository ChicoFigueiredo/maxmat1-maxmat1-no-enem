\section{Questão 161 - Estatística, Média Ponderada}

Em uma fábrica de refrigerantes, é necessário que se faça periodicamente o controle no processo de engarrafamento para evitar que sejam envasadas garrafas fora da  especificação do volume escrito no rótulo.

Diariamente, durante 60 dias, foram anotadas as quantidades de garrafas fora dessas especificações. O resultado está apresentado no quadro.


% https://www.tablesgenerator.com - Please add the following required packages to your document preamble:
% \usepackage[table,xcdraw]{xcolor}
% If you use beamer only pass "xcolor=table" option, i.e. \documentclass[xcolor=table]{beamer}
\begin{center}
    \begin{tabular}{cc}
        \rowcolor[HTML]{BFBFBF} 
        {\color[HTML]{000000} \begin{tabular}[c]{@{}c@{}}Quantidade de garrafas \\ fora das especificações \\ por dia\end{tabular}} & {\color[HTML]{000000} \begin{tabular}[c]{@{}c@{}}Quantidade \\ de \\ dias\end{tabular}} \\
        {\color[HTML]{000000} 0}                                                                                                    & {\color[HTML]{000000} 52}                                                               \\
        {\color[HTML]{000000} 1}                                                                                                    & {\color[HTML]{000000} 5}                                                                \\
        {\color[HTML]{000000} 2}                                                                                                    & {\color[HTML]{000000} 2}                                                                \\
        {\color[HTML]{000000} 3}                                                                                                    & {\color[HTML]{000000} 1}                                                               
    \end{tabular}
\end{center}


A média diária de garrafas fora das especificações no período considerado é

(A)  0,1.

(B)  0,2.

(C)  1,5.

(D)  2,0.

(E)  3,0.

\textbf{Resolução}

Interpetando adequadamente a tabela e a questão trata-se de uma média ponderada 

% https://www.tablesgenerator.com - Please add the following required packages to your document preamble:
% \usepackage[table,xcdraw]{xcolor}
% If you use beamer only pass "xcolor=table" option, i.e. \documentclass[xcolor=table]{beamer}
\begin{center}
    \begin{tabular}{ccc}
        \rowcolor[HTML]{BFBFBF} 
        {\color[HTML]{000000} \begin{tabular}[c]{@{}c@{}}Quantidade de garrafas \\ fora das especificações \\ por dia \\ (1) \end{tabular}} & {\color[HTML]{000000} \begin{tabular}[c]{@{}c@{}}Quantidade \\ de \\ dias \\ (2) \end{tabular}} & {\color[HTML]{000000} \begin{tabular}[c]{@{}c@{}} (1) $\times$ (2) \end{tabular}} \\
        {\color[HTML]{000000} 0} & {\color[HTML]{000000} 52}  & {\color[HTML]{000000} 0} \\
        {\color[HTML]{000000} 1} & {\color[HTML]{000000} 5}   & {\color[HTML]{000000} 5}  \\
        {\color[HTML]{000000} 2} & {\color[HTML]{000000} 2}   & {\color[HTML]{000000} 4}  \\
        {\color[HTML]{000000} 3} & {\color[HTML]{000000} 1}   & {\color[HTML]{000000} 3}  \\         {\color[HTML]{000000} Soma} & {\color[HTML]{000000} 60}   & {\color[HTML]{000000} 12}  \\ 
    \end{tabular}
\end{center}

Matematicamente falando



\begin{eqnarray*}
    M_{p} 	&=& \dfrac{0 \cdot 52 + 1 \cdot 5 + 2 \cdot 2 + 3 \cdot 1}{52 + 5 + 2 + 1} \\
            &=& \dfrac{0 + 5 + 4 + 3}{60} \\
            &=& \dfrac{12}{60} \\
            &=& \dfrac{1}{5} \\
    M_{p}   &=& 0,2 \\
\end{eqnarray*}


\textbf{Rascunho}

\opdiv[decimalsepsymbol={,},displayintermediary=all]{12}{60}\flexquad{3}
\opdiv[decimalsepsymbol={,},displayintermediary=all]{1}{5}\flexquad{3}

\begin{center}
    \href{https://youtu.be/}{
        \qrcode{https://youtu.be/}
    }\\
    Resolução: \url{https://youtu.be/}
\end{center}
