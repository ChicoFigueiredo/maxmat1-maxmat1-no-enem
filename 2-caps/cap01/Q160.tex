\section{Questão 160 - Estatística, Média Aritmética}

O preparador físico de um time de basquete dispõe de um plantel de 20 jogadores, com média de altura igual a 1,80 m. No último treino antes da estreia em um campeonato, um dos jogadores desfalcou o time em razão de uma séria contusão, forçando o técnico a contratar outro jogador para recompor o grupo.

Se o novo jogador é 0,20 m mais baixo que o anterior, qual é a média de altura, em metro, do novo grupo?

(A)  1,60

(B)  1,78

(C)  1,79

(D)  1,81

(E)  1,82

\textbf{Resolução}

Dado que média $ M = \dfrac{S_{20}}{n} = \dfrac{\Sigma_{1}^{n} J_{i}}{n} \Rightarrow M = \dfrac{\Sigma_{1}^{20} J_{i}}{20} = 1,80m $

Donde que se conclui que a soma das alturas dos jogadores é $ S_{20} = \Sigma_{1}^{20} J_{i} = 1,80 \cdot 20 \Rightarrow \Sigma_{1}^{20} J_{i} = 36m $

A nova soma, com o jogador trocado, diminuiu 0,20 m, ou seja

$ S'_{20} = S_{20} - 0,20 \Rightarrow S'_{20} = 36m - 0,20m  \Rightarrow S'_{20} = 35,8m $

A nova média é $ M' = \dfrac{S'_{20}}{20} \Rightarrow  M' = \dfrac{35,8}{20} \Rightarrow  M' = 1,79 $

\textbf{Outra Resolução}

O leitor pode também resolver rapidamente verificando que a nova média será decrescida de 0,20 m dividido pelos 20 membros, ou seia $ \dfrac{0,20}{20} = 0,01m $ ou seja, caindo de $ 1,80m $ para $ 1,80 - 0,01m = 1,79m $ 

\textbf{Resposta:} A nova média é 1,79m, correspondendo ao gabarito (C)

\textbf{Rascunho}

\opmul[decimalsepsymbol={,},displayintermediary=all]{1,80}{20}\flexquad{3}
\opsub[decimalsepsymbol={,},displayintermediary=all]{36}{0,20}\flexquad{3}
\opdiv[decimalsepsymbol={,},displayintermediary=all]{35,8}{20}\flexquad{3}

\opdiv[decimalsepsymbol={,},displayintermediary=all]{0,20}{20}\flexquad{3}
\opsub[decimalsepsymbol={,},displayintermediary=all]{1,80}{0,01}\flexquad{3}



\begin{center}
    \href{https://youtu.be/}{
        \qrcode{https://youtu.be/}
    }\\
    Resolução: \url{https://youtu.be/}
\end{center}
