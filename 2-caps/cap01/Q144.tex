\section{Questão 144 - Matriz, Interpretação de tabelas}

Um professor aplica, durante os cinco dias úteis de uma semana, testes com quatro questões de múltipla escolha a cinco alunos. Os resultados foram representados na matriz.
\[ 
    \left[ 
        \begin{matrix}
           3 & 2 & 0 & 1 & 2 \\
           3 & 2 & 4 & 1 & 2 \\
           2 & 2 & 2 & 3 & 2 \\
           3 & 2 & 4 & 1 & 0 \\
           0 & 2 & 0 & 4 & 4
        \end{matrix}
    \right] 
 \]
Nessa matriz os elementos das linhas de 1 a 5 representam as quantidades de questões acertadas pelos alunos Ana, Bruno, Carlos, Denis e Érica, respectivamente, enquanto que as colunas de 1 a 5 indicam os dias da semana, de segunda-feira a sexta-feira, respectivamente, em que os testes foram aplicados.

O teste que apresentou maior quantidade de acertos foi o aplicado na

(A)  segunda-feira.

(B)  terça-feira.

(C)  quarta-feira.

(D) quinta-feira.

(E)  sexta-feira.

\textbf{Resolução}

Basta interpretar a matriz em que cada linha representa um dos alunos e a coluna a quantidade de acertos no dia da semana. 

A coluna que tiver o maior somatório será a coluna do dia da semana que teve mais acertos.

\newcommand{\myunit}{1 cm}
\newcommand{\nodeRectColumn}[6]{
    \draw [ color={#1}  ,draw opacity=1 ] (A-#2-#4.north west) -- (A-#2-#5.north east) -- (A-#3-#5.south east) -- (A-#3-#4.south west) -- cycle;
    \node [ color={#1}  ,below = of A-#3-#4.south] (A-#3-#4-s) {#6};
    \draw [ color={#1}  ,draw opacity=1, -> ]   (A-#3-#4.south) -- (A-#3-#4-s) ; 
}

\newcommand{\LinhaExplicativa}[3] {
    \node [ color={#1}  ,left = of A-#2-1.west] (A-#2-1-l) {#3}; \draw [  color={#1}  , draw opacity=1, -> ]   (A-#2-1-l.east) -- (A-#2-1.west) ; 
}

\newcommand{\ColunaExplicativa}[3] {
    \node [ color={#1}  ,above = of A-1-#2.north] (A-1-#2-c) {#3}; \draw [  color={#1}  , draw opacity=1, -> ]   (A-1-#2-c.south) -- (A-1-#2.north) ; 
}

\tikzset{
    node style sp/.style={draw,circle,minimum size=\myunit},
    node style ge/.style={circle,minimum size=\myunit},
    arrow style mul/.style={draw,sloped,midway,fill=white},
    arrow style plus/.style={midway,sloped,fill=white},
}
\noindent \begin{tikzpicture}[>=latex]
% les matrices
\matrix (A) [matrix of math nodes,%
nodes = {node style ge},%
left delimiter  = {[},%
right delimiter = {]}] at (0,0)
{%
    3 & 2 & 0 & 1 & 2 \\
    3 & 2 & 4 & 1 & 2 \\
    2 & 2 & 2 & 3 & 2 \\
    3 & 2 & 4 & 1 & 0 \\
    0 & 2 & 0 & 4 & 4 \\
};

\nodeRectColumn{red}{1}{5}{1}{1}{11}
\nodeRectColumn{rgb, 100:red, 75; green, 75; blue, 75}{1}{5}{2}{2}{10}
\nodeRectColumn{rgb, 100:red, 75; green, 75; blue, 75}{1}{5}{3}{3}{10}
\nodeRectColumn{rgb, 100:red, 75; green, 75; blue, 75}{1}{5}{4}{4}{10}
\nodeRectColumn{rgb, 100:red, 75; green, 75; blue, 75}{1}{5}{5}{5}{10}

\LinhaExplicativa{rgb, 100:red, 50; green, 50; blue, 50}{1}{Ana}
\LinhaExplicativa{rgb, 100:red, 50; green, 50; blue, 50}{2}{Bruno}
\LinhaExplicativa{rgb, 100:red, 50; green, 50; blue, 50}{3}{Carlos}
\LinhaExplicativa{rgb, 100:red, 50; green, 50; blue, 50}{4}{Denis}
\LinhaExplicativa{rgb, 100:red, 50; green, 50; blue, 50}{5}{Érica}

\ColunaExplicativa{rgb, 100:red, 50; green, 50; blue, 50}{1}{Seg}
\ColunaExplicativa{rgb, 100:red, 50; green, 50; blue, 50}{2}{Ter}
\ColunaExplicativa{rgb, 100:red, 50; green, 50; blue, 50}{3}{Qua}
\ColunaExplicativa{rgb, 100:red, 50; green, 50; blue, 50}{4}{Qui}
\ColunaExplicativa{rgb, 100:red, 50; green, 50; blue, 50}{5}{Sex}

\end{tikzpicture}

\textbf{Resposta:} Vemos claramente que o dia da semana que possui o maior somatório é a coluna 1, correspondente a da segunda feira. O que corresponde ao gabarito (A)

\begin{center}
    \href{https://youtu.be/lLzEawlb-xw}{
        \qrcode{https://youtu.be/lLzEawlb-xw}
    }\\
    Resolução: \url{https://youtu.be/lLzEawlb-xw}
\end{center}
