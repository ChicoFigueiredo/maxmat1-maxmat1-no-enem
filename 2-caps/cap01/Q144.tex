\section{Questão 144 - Matriz, Interpretação de tabelas}

Um professor aplica, durante os cinco dias úteis de uma semana, testes com quatro questões de múltipla escolha a cinco alunos. Os resultados foram representados na matriz.
\[ 
    \left[ 
        \begin{matrix}
           3 & 2 & 0 & 1 & 2 \\
           3 & 2 & 4 & 1 & 2 \\
           2 & 2 & 2 & 3 & 2 \\
           3 & 2 & 4 & 1 & 0 \\
           0 & 2 & 0 & 4 & 4
        \end{matrix}
    \right] 
 \]
Nessa matriz os elementos das linhas de 1 a 5 representam as quantidades de questões acertadas pelos alunos Ana, Bruno, Carlos, Denis e Érica, respectivamente, enquanto que as colunas de 1 a 5 indicam os dias da semana, de segunda-feira a sexta-feira, respectivamente, em que os testes foram aplicados.

O teste que apresentou maior quantidade de acertos foi o aplicado na

(A)  segunda-feira.

(B)  terça-feira.

(C)  quarta-feira.

(D) quinta-feira.

(E)  sexta-feira.

\textbf{Resolução}

\textbf{Rascunho}

\opmul[decimalsepsymbol={,},displayintermediary=all]{1.01}{1.01}\flexquad{3}
\opdiv[decimalsepsymbol={,},displayintermediary=all]{202}{1.01}\flexquad{3}

\opset{strikedecimalsepsymbol={\rlap{,}\rule[-1pt]{3pt}{0.4pt}}}
\opdiv[decimalsepsymbol={,},shiftdecimalsep=both,displayintermediary=all]{204.02}{1.0201}\quad


\begin{center}
    \href{https://youtu.be/}{
        \qrcode{https://youtu.be/}
    }\\
    Resolução: \url{https://youtu.be/}
\end{center}
