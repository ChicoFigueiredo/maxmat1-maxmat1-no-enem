\section{Questão 144 - Matriz, Interpretação de tabelas}

Um professor aplica, durante os cinco dias úteis de uma semana, testes com quatro questões de múltipla escolha a cinco alunos. Os resultados foram representados na matriz.
\[ 
    \left[ 
        \begin{matrix}
           3 & 2 & 0 & 1 & 2 \\
           3 & 2 & 4 & 1 & 2 \\
           2 & 2 & 2 & 3 & 2 \\
           3 & 2 & 4 & 1 & 0 \\
           0 & 2 & 0 & 4 & 4
        \end{matrix}
    \right] 
 \]
Nessa matriz os elementos das linhas de 1 a 5 representam as quantidades de questões acertadas pelos alunos Ana, Bruno, Carlos, Denis e Érica, respectivamente, enquanto que as colunas de 1 a 5 indicam os dias da semana, de segunda-feira a sexta-feira, respectivamente, em que os testes foram aplicados.

O teste que apresentou maior quantidade de acertos foi o aplicado na

(A)  segunda-feira.

(B)  terça-feira.

(C)  quarta-feira.

(D) quinta-feira.

(E)  sexta-feira.

\textbf{Resolução}

\newcommand{\myunit}{1 cm}
\tikzset{
    node style sp/.style={draw,circle,minimum size=\myunit},
    node style ge/.style={circle,minimum size=\myunit},
    arrow style mul/.style={draw,sloped,midway,fill=white},
    arrow style plus/.style={midway,sloped,fill=white},
}
\begin{tikzpicture}[>=latex]
% les matrices
\matrix (A) [matrix of math nodes,%
             nodes = {node style ge},%
             left delimiter  = {[},%
             right delimiter = {]}] at (0,0)
{%
    3 & 2 & 0 & 1 & 2 \\
    3 & 2 & 4 & 1 & 2 \\
    2 & 2 & 2 & 3 & 2 \\
    3 & 2 & 4 & 1 & 0 \\
    0 & 2 & 0 & 4 & 4 \\
};

\draw (A-1-1.north west) -- (A-1-1.north east) -- (A-5-1.south east) -- (A-5-1.south west) -- cycle;

\end{tikzpicture}


\begin{center}
    \href{https://youtu.be/lLzEawlb-xw}{
        \qrcode{https://youtu.be/lLzEawlb-xw}
    }\\
    Resolução: \url{https://youtu.be/lLzEawlb-xw}
\end{center}
