\section{Questão 155 - Sequência, Resto da Divisão}

Após o Fórum Nacional Contra a Pirataria (FNCP) incluir a linha de autopeças em campanha veiculada contra a falsificação, as agências fiscalizadoras divulgaram que os cinco principais produtos de autopeças falsificados são: rolamento, pastilha de freio, caixa de direção, catalisador e amortecedor.

Após uma grande apreensão, as peças falsas foram cadastradas utilizando-se a codificação:

\begin{flushright}
    {\tiny Disponível em: www.oficinabrasil.com.br. Acesso em: 25 ago. 2014 (adaptado).}
\end{flushright}

1: rolamento, 2: pastilha de freio, 3: caixa de direção, 4: catalisador e 5: amortecedor.

Ao final obteve-se a sequência: 5, 4, 3, 2, 1, 2, 3, 4, 5, 4, 3, 2, 1, 2, 3, 4, 5, 4, 3, 2, 1, 2, 3, 4, ... que apresenta um padrão de formação que consiste na repetição de um bloco de números. Essa sequência descreve a ordem em que os produtos apreendidos foram cadastrados. 

O 2 015º item cadastrado foi um(a)

(A) rolamento.

(B) catalisador.

(C) amortecedor.

(D) pastilha de freio.

(E) caixa de direção.


\textbf{Resolução}

Podemos verificar que a lista possui um padrão de repetição de tamanho de 8 itens conforme esquema abaixo:

\renewcommand{\myunit}{.4cm}
\newcommand{\nodeRectL}[7]{
    \draw [ color={#2}  ,draw opacity=1 ] (#1-#3-#5.north west) -- (#1-#3-#6.north east) -- (#1-#4-#6.south east) -- (#1-#4-#5.south west) -- cycle;
}

\tikzset{
    node style sp/.style={draw,circle,minimum size=\myunit},
    node style ge/.style={circle,minimum size=\myunit},
    arrow style mul/.style={draw,sloped,midway,fill=white},
    arrow style plus/.style={midway,sloped,fill=white},
}

\noindent \resizebox{\linewidth}{!}{
    \begin{tikzpicture}[scale=.6]
        
        \matrix (A) [matrix of math nodes,%
        nodes = {node style ge},ampersand replacement=\&] at (0,0)
        {%
            5 \& 4 \& 3 \& 2 \& 1 \& 2 \& 3 \& 4 \& 5 \& 4 \& 3 \& 2 \& 1 \& 2 \& 3 \& 4 \& 5 \& 4 \& 3 \& 2 \& 1 \& 2 \& 3 \& 4  \& 5 \& 4 \& 3 \& ... \\
        };
        
        
        \matrix (B) [matrix of math nodes,%
        nodes = {node style ge},ampersand replacement=\&] at (0,-1)
        {%
            5 \& 4 \& 3 \& 2 \& 1 \& 2 \& 3 \& 4 \& 5 \& 4 \& 3 \& 2 \& 1 \& 2 \& 3 \& 4 \& 5 \& 4 \& 3 \& 2 \& 1 \& 2 \& 3 \& 4  \& 5 \& 4 \& 3 \& ... \\
        };	
        
        \nodeRectL{B}{red}{1}{1}{1}{8};
        \nodeRectL{B}{red}{1}{1}{9}{16};
        \nodeRectL{B}{red}{1}{1}{17}{24};
        
    \end{tikzpicture}
}

A resolução dessa questão passa pelo conhecimento do algoritmo de divisão, mais precisamente do resto...

Primeiro faremos a divisão do numeral 2.015 por 8 que é o tamanho da sequência 

\opidiv[decimalsepsymbol={,},displayintermediary=all]{2015}{8}\flexquad{3}

Ou seja, teremos a repetição de 251 vezes a sequência completa (5, 4, 3, 2, 1, 2, 3, 4) e mais uma sequência incompleta com 7 elementos (5, 4, 3, 2, 1, 2, 3), ou seja, o 2.015º elemento de nossa sequência completa será o 7º elemento o valor 3, que corresponde, conforme de-para, \textbf{caixa de direção}, o que corresponde ao gabarito (E) 

\textbf{Rascunho}

\opmul[decimalsepsymbol={,},displayintermediary=all]{1.01}{1.01}\flexquad{3}
\opset{strikedecimalsepsymbol={\rlap{,}\rule[-1pt]{3pt}{0.4pt}}}
\opdiv[decimalsepsymbol={,},shiftdecimalsep=both,displayintermediary=all]{204.02}{1.0201}\quad


\begin{center}
    \href{https://youtu.be/}{
        \qrcode{https://youtu.be/}
    }\\
    Resolução: \url{https://youtu.be/}
\end{center}
