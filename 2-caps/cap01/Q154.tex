\section{Questão 154 - Logaritmos, unidades de medida, interpretação tabela}

\begin{wrapfigure}{r}{0.2\textwidth}
    \begin{center}
        \qrcode{https://youtu.be/S22SHYt4n-o}
    \end{center}
\end{wrapfigure}

Charles Richter e Beno Gutenberg desenvolveram a escala Richter, que mede a magnitude de um terremoto. Essa escala pode variar de 0 a 10, com possibilidades de valores maiores. O quadro mostra a escala de magnitude local $ (M_{s}) $ de um terremoto que é utilizada para descrevê-lo.


% Please add the following required packages to your document preamble:
% \usepackage[table,xcdraw]{xcolor}
% If you use beamer only pass "xcolor=table" option, i.e. \documentclass[xcolor=table]{beamer}
% https://www.tablesgenerator.com
\begin{center}
    \begin{tabular}{cc}
        \rowcolor[HTML]{BFBFBF} 
        {\color[HTML]{000000} Descrição} & {\color[HTML]{000000}
            \begin{tabular}[c]{@{}c@{}}
                Magnitude   local $ (M_{s}) $\\  $ (\mu m \cdot Hz) $\end{tabular}} \\
                {\color[HTML]{000000} Pequeno}   & {\color[HTML]{000000} $ 0 \leqslant M_{s} \leqslant 3,9 $}                                                                \\
                {\color[HTML]{000000} Ligeiro}   & {\color[HTML]{000000} $ 4,0 \leqslant M_{s} \leqslant 4,9 $}                                                              \\
                {\color[HTML]{000000} Moderado}  & {\color[HTML]{000000} $ 5,0 \leqslant M_{s} \leqslant 5,9 $}                                                              \\
                {\color[HTML]{000000} Grande}    & {\color[HTML]{000000} $ 6,0 \leqslant M_{s} \leqslant 9,9 $}                                                              \\
                {\color[HTML]{000000} Extremo}   & {\color[HTML]{000000} $ M_{s} \geqslant   10,0 $}                                                                
    \end{tabular}
\end{center}



Para se calcular a magnitude local, usa-se a fórmula $ M_{s} = 3,30 + log(A \cdot f)$, em que $ A $ representa a amplitude máxima da onda registrada por um sismógrafo em micrômetro ($\mu m$) e $ f $ representa a frequência da onda, em hertz $ (Hz) $. Ocorreu um terremoto com amplitude máxima de $ 2 000 \mu m $ e frequência de $ 0,2 Hz $. 

{\scriptsize 
    \begin{flushright}
        Disponível em: http://cejarj.cecierj.edu.br. Acesso em: 1 fev. 2015 (adaptado). 
    \end{flushright}
}

Utilize 0,3 como aproximação para log 2. 

De acordo com os dados fornecidos, o terremoto ocorrido pode ser descrito como

(A) Pequeno.

(B) Ligeiro.

(C) Moderado.

(D) Grande.

(E) Extremo.



\textbf{Resolução}

Temos um caso simples de substituição de variáveis, e bom uso das propriedades de logaritmos:


\begin{eqnarray*}
    M_{s} 	&=& 3,30 + log(A \cdot f) \\
            &=& 3,30 + log(2000 \cdot 0,2) \\
            &=& 3,30 + log(400) \\
            &=& 3,30 + log(4 \cdot 100) \\
            &=& 3,30 + log(2^{2} \cdot 10^{2}) \\
            &=& 3,30 + log(2^{2}) + log(10^{2}) \\
            &=& 3,30 + 2 \cdot log(2) + 2 \cdot log(10) \\
            &=& 3,30 + 2 \cdot 0,3 + 2 \cdot 1 \\
            &=& 3,30 + 0,6 + 2 \\
      M_{s} &=& 5,9 \\
\end{eqnarray*}


\textbf{Rascunho}

\opmul[decimalsepsymbol={,},displayintermediary=all]{2000}{0.2}\flexquad{5}
\opmul[decimalsepsymbol={,}]{0.3}{2}\flexquad{5}
\opadd[decimalsepsymbol={,},displayintermediary=all]{3.3}{0.6}{2}\qquad




