\section{Questão 156 - Combinatória}

Durante suas férias, oito amigos, dos quais dois são canhotos, decidem realizar um torneio de vôlei de raia.
Eles precisam formar quatro duplas para a realização do torneio. Nenhuma dupla pode ser formada por dois jogadores canhotos.

De quantas maneiras diferentes podem ser formadas essas quatro duplas?

\noindent (A)  69

\noindent (B)  70

\noindent (C)  90

\noindent (D)  104

\noindent (E) 105

\textbf{Resolução}
Podemos usar o raciocínio destrutivo: Montar todas as possíveis subtraindo todas as que não nos interessam:

\renewcommand{\CancelColor}{\color{red}}

Formação de todas as combinações possíveis, sem levar em conta os canhotos: 
\begin{eqnarray*}
\dfrac{C_{8, 2} \cdot C_{6, 2} \cdot C_{4, 2} \cdot C_{2, 2}  }{4!} &=& \\
C_{8, 2} \cdot C_{6, 2} \cdot C_{4, 2} \cdot C_{2, 2}  \cdot \dfrac{1}{4!} &=& \\
\dfrac{8!}{2! 6!} \cdot \dfrac{6!}{2! 4!} \cdot \dfrac{4!}{2! 2!} \cdot \dfrac{2!}{1! 0!} \cdot \dfrac{1}{4!}  &=& \\
\dfrac{8!}{2! \cancel{6!}} \cdot \dfrac{\cancel{6!}}{2! \cancel{4!}} \cdot \dfrac{\cancel{4!}}{2! 2!} \cdot \cancelto{1}{\dfrac{2!}{1! 0!}} \cdot \dfrac{1}{4!}  &=& \\
\dfrac{8 \cdot 7 \cdot \cancelto{3}{6} \cdot 5 \cdot 4!}{2 \cdot 2 \cdot 2 \cdot 2 \cdot 4!}  &=& \\
\dfrac{\bcancel{8} \cdot 7 \cdot 6 \cdot 5 \cdot \cancel{4!}}{\bcancel{2} \cdot \bcancel{2} \cdot \cancel{2} \cdot \bcancel{2} \cdot \cancel{4!}}  &=& \\
7 \cdot 3 \cdot 5  &=& 105\\
\end{eqnarray*}

\textbf{Outra resolução:}
Vamos por etapas:

\begin{enumerate}
    \item A primeira dupla pode escolher 1 dos dois canhotos e o outro jogador obrigatoriamente será destro. Isto posto, temos 2 formas de escolher o canhoto e 6 maneiras de escolher o destro, ou seja, temos $ 2 \cdot 6 = 12 $ duplas possíveis. 
    \item A segunda dupla só pode escolher 1 entre os canhotos disponíveis e 5 entre os destros disponíveis, ou seja $ 1 \cdot 5  = 5 $ maneiras distintas.
    \item A terceira dupla só pode ser montada com destros. ou seja podemos escolher de 4 maneiras possíveis o primeiro integrante e 3 maneiras possíveis o segundo membro, ou seja $ 4 \cdot 3  = 12 $ maneiras. Como a ordem não importa, dividimos por 2 as escolhas $ \dfrac{12}{2} = 6$ duplas distintas
    \item Como só sobra 2 integrantes, ele formam a única dupla
\end{enumerate}

Contabilizando, temos que por conta da natureza combinatória dos 2 primeiros grupos, tenho que dividir a multiplicação deles por 2, bem como os dois últimos grupos também preciso dividir por 2.

Temos:

\[
\dfrac{12 \cdot 5}{2} \cdot \dfrac{6 \cdot 1}{2} = 6 \cdot 5 \cdot 3 \cdot 1 = 30 \cdot 3 = 90
\]

O que corresponde ao gabarito (C)

\begin{center}
    \href{https://www.youtube.com/watch?v=s-7p8Bi8TE4}{
        \qrcode{https://www.youtube.com/watch?v=s-7p8Bi8TE4}
    }\\
    Resolução: \url{https://www.youtube.com/watch?v=s-7p8Bi8TE4}
\end{center}
