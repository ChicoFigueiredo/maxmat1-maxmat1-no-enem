\section{Questão 137 - Logaritmo e Inequação}

A \textit{Hydrangea macrophylla} é uma planta com flor azul ou cor-de-rosa, dependendo do pH do solo no qual está plantada. Em solo ácido (ou seja, com $ pH < 7 $) a flor é azul, enquanto que em solo alcalino (ou seja, com $ pH > 7 $) a flor é rosa. Considere que a \textit{Hydrangea} cor-de-rosa mais valorizada comercialmente numa determinada região seja aquela produzida em solo com pH inferior a 8. 
Sabe-se que $ pH = - log_{10} x $, em que $ x $ é a concentração de íon hidrogênio ($ H^+ $).
Para produzir a Hydrangea cor-de-rosa de maior valor comercial, deve-se preparar o solo de modo que x assuma

\noindent (A) qualquer valor acima de $ 10^{-8} $.

\noindent (B) qualquer valor positivo inferior a $ 10^{-7} $.

\noindent (C) valores maiores que 7 e menores que 8.

\noindent (D) valores maiores que 70 e menores que 80.

\noindent (E) valores maiores que $ 10^{-8} $ e menores que $ 10^{-7} $.


\textbf{Resolução}

O problema apresenta 2 condições:

\noindent \textbf{Condição 1: Hydrangea cor-de-rosa $ \Rightarrow pH > 7 $ }
\begin{align*}
            pH &> 7        &  \\
    -log_{10}x &> 7        & \cdot (-1) \\
     log_{10}x &> -7       & 10^e  \\
10^{log_{10}x} &> 10^{-7}  & 10^{log_{10}x} = x\\
            x  &> -7       & 
\end{align*}

\textbf{Rascunho}

\opmul[decimalsepsymbol={,},displayintermediary=all]{1.01}{1.01}\flexquad{3}
\opdiv[decimalsepsymbol={,},displayintermediary=all]{202}{1.01}\flexquad{3}
\opset{strikedecimalsepsymbol={\rlap{,}\rule[-1pt]{3pt}{0.4pt}}}
\opdiv[decimalsepsymbol={,},shiftdecimalsep=both,displayintermediary=all]{204.02}{1.0201}\quad


\begin{center}
    \href{https://youtu.be/}{
        \qrcode{https://youtu.be/}
    }\\
    Resolução: \url{https://youtu.be/}
\end{center}
