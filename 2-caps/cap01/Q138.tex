\section{Questão 138 - Porcentagem}

Uma pessoa, que perdeu um objeto pessoal quando visitou uma cidade, pretende divulgar nos meios de comunicação informações a respeito da perda desse objeto e de seu contato para eventual devolução. 
No entanto, ela lembra que, de acordo com o Art. 1 234 do Código Civil, poderá ter que pagar pelas despesas do transporte desse objeto até sua cidade e poderá ter que recompensar a pessoa que lhe restituir o objeto em, pelo menos, 5\% do valor do objeto. 

Ela sabe que o custo com transporte será de um quinto do valor atual do objeto e, como ela tem muito interesse em reavê-lo, pretende ofertar o maior percentual possível de recompensa, desde que o gasto total com as despesas não ultrapasse o valor atual do objeto.

Nessas condições, o percentual sobre o valor do objeto, dado como recompensa, que ela deverá ofertar é igual a

(A) 20\%

(B) 25\%

(C) 40\%

(D) 60\%

(E) 80\%

\textbf{Resolução}

$ V $ Valor do objeto

$ \dfrac{1}{5}V $ será o valor do trans porte, vale notar que $ \dfrac{1}{5} =  \dfrac{1 \cdot 20}{5 \cdot 20} = \dfrac{20}{100} = 0,2 = 20\% \Rightarrow 20\%V $

A recompensa será o que o total do valor do objeto menos o transporte, ou seja


\begin{eqnarray*}
    X &=& V - \dfrac{1}{5}V \\
      &=& \left(1 - \dfrac{1}{5}\right)V \\
      &=& \left(\dfrac{5 - 1}{5}\right)V \\
      &=& \dfrac{4}{5}V \\
      &=& 0,8V \\
      &=& 80\%V 
\end{eqnarray*}

\textbf{Resposta:} O percentual da recompensa será de 80\% do valor do objeto

\textbf{Rascunho}

\opdiv[decimalsepsymbol={,},shiftdecimalsep=both,displayintermediary=all]{1}{5}\flexquad{2}
\opdiv[decimalsepsymbol={,},shiftdecimalsep=both,displayintermediary=all]{20}{100}\flexquad{2}
\opdiv[decimalsepsymbol={,},shiftdecimalsep=both,displayintermediary=all]{4}{5}\flexquad{2}


\begin{center}
    \href{https://youtu.be/hRFaaCCGBQo}{
        \qrcode{https://youtu.be/hRFaaCCGBQo}
    }\\
    Resolução: \url{https://youtu.be/hRFaaCCGBQo}
\end{center}
